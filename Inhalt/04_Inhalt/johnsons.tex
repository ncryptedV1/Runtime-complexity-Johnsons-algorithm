\chapter{Johnson's algorithm}
Johnson's algorithm combines the previously considered algorithms Dijkstra \& Bellman-Ford to solve the \ac{APSP} with a faster runtime complexity than Floyd-Warshall algorithm, which runs in $O(|V|^3)$. \cite{Kiruthika.2012, Tamimi.2015}

To do so the limitations of Dijkstra have to be considered. As Dijkstra doesn't support negative edge weights Bellman-Ford is used to transform the graph into a graph with non-negative edge weights (or raises detected negative cycles). After that Dijkstra is applied to every vertex in the graph to solve the \ac{SSSP} for every vertex. The final shortest distances can be computed by a combination of the values computed by Bellman-Ford and Dijkstra. \cite{Kiruthika.2012}

The runtime complexity of this algorithm consists of the different stages of the algorithm. The first stage - transformation via Bellman-Ford - is the plain application of the Bellman-Ford algorithm and therefore has a runtime complexity of $O(|V|*|E|)$. The second stage - computation of shortest paths for all pairs of vertices via Dijkstra - adds a factor of $|V|$ to the runtime complexity as Dijkstra is applied to every vertex. Therefore this stage has a runtime complexity of $O(|V|^2*log|V|)$. The last stage can be computed in constant time as it just accesses previously computed values. \cite{Kiruthika.2012, Tamimi.2015} \\
Putting together the single parts Johnson's algorithm exhibits a runtime complexity of $O(|V|^2*log|V| + |V|*|E|)$, which is considerably faster than $O(|V|^3)$. Therefore we have found a faster alternative to Floyd-Warshall algorithm.