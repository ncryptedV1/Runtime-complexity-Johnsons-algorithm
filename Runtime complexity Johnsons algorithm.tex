% ---- Präambel mit Angaben zum Dokument
\input{Inhalt/00_Latex/praeambel}

% ---- Elektronische Version oder Gedruckte Version?
% ---- Unterschied: Die elektronische Version enthält keinen Platzhalter für die Unterschrift
\usepackage{ifthen}
\newboolean{e-Abgabe}
\setboolean{e-Abgabe}{true}    % false=gedruckte Fassung

% ---- Persönlichen Daten:
\newcommand{\titel}{Runtime complexity analysis of Johnson's algorithm}
\newcommand{\titelheader}{Runtime complexity analysis of Johnson's algorithm}
\newcommand{\arbeit}{Exercise as part of the lecture Wissenschaftliches Arbeiten I}
\newcommand{\studiengang}{Informatik}
\newcommand{\studienjahr}{2021}
\newcommand{\autor}{Oliver Schirmer}
\newcommand{\autorReverse}{Schirmer, Oliver}
\newcommand{\verfassungsort}{Karlsruhe}
\newcommand{\matrikelnr}{7059763}
\newcommand{\kurs}{TINF20B2}
\newcommand{\bearbeitungsmonat}{Juli 2021}
\newcommand{\abgabe}{20. Juli 2021}
\newcommand{\bearbeitungszeitraum}{13.07.2021 - 20.07.2021}
\newcommand{\firmaName}{SAP SE}
\newcommand{\firmaStrasse}{Dietmar-Hopp-Allee 16}
\newcommand{\firmaPlz}{69190 Walldorf, Deutschland}
\newcommand{\betreuerFirma}{Simon Nichterlein}
\newcommand{\betreuerDhbw}{Heinrich Braun}

\input{Inhalt/00_Latex/kopfundFusszeile}

% ---- Hilfreiches
\newcommand{\zB}{z.\,B.}   % "z.B." mit kleinem Leeraum dazwischen (ohne wäre nicht korrekt)
\newcommand{\ZB}{Z.\,B.}
\newcommand{\dash}{d.\,h.}
\newcommand{\Dash}{D.\,h.}
\newcommand{\mH}{m.\,H.}
\newcommand{\MH}{M.\,H.}

\newcommand{\code}[1]{\texttt{#1}} % Ist einfacher zu schreiben als ständig \texttt und erlaubt
                                   % Änderungen im Nachhinein, wenn man z.B. Inline-Code anders stylen möchte.

% ---- Silbentrennung (falls LaTeX defaults falsch / nicht gewünscht sind)
\hyphenation{HANA}         % anstatt HA-NA
\hyphenation{Graph-Script} % anstatt GraphS-cript

% ---- Beginn des Dokuments
\begin{document}
\setlength{\parindent}{0pt}              % Keine Paragraphen Einrückung.
% Dafür haben wir den Abstand zwischen den Paragraphen.
\setcounter{secnumdepth}{2}              % Nummerierungstiefe fürs Inhaltsverzeichnis
\setcounter{tocdepth}{2}                 % Tiefe des Inhaltsverzeichnisses. Ggf. so anpassen,
% dass das Verzeichnis auf eine Seite passt.
\sffamily                                % Serifenlose Schrift verwenden.

% ---- Vorspann
% ------ Titelseite
\singlespacing
\thispagestyle{empty}
\begin{titlepage}
	\enlargethispage{4cm}

	\begin{figure}           % Logo vom Ausbildungsbetrieb und der DHBW
		% \vspace*{-5mm} % Sollte dein Titel zu lang werden, kannst du mit diesem "Hack" 
		%                  den Inhalt der Seite nach oben schieben.
		\begin{minipage}{0.49\textwidth}
			\flushleft
			\includegraphics[height=2.5cm]{Bilder/Logos/Logo_DHBW.pdf}
		\end{minipage}
	\end{figure}
	\vspace*{0.1cm}

	\begin{center}
		\huge{\textbf{\titel}}\\[1.5cm]
		\Large{\textbf{\arbeit}}\\[0.5cm]
		\normalsize{as part of the exam for\\[1ex] \textbf{Bachelor of Science (B.Sc.)}}\\[0.5cm]
		\Large{of the course \studiengang}\\[1ex]
		\normalsize{at the Dualen Hochschule Baden-Württemberg Karlsruhe}\\[1cm]
		\normalsize{from}\\[1ex] \Large{\textbf{\autor}} \\[1cm]
	\end{center}

	\begin{center}
		\vfill
		\begin{tabular}{ll}
			Submission date:                        & \abgabe               \\[0.2cm]
			Processing period:                      & \bearbeitungszeitraum \\[0.2cm]
			Matriculation number, class:            & \matrikelnr , \kurs   \\[0.2cm]
			Training company:                       & \firmaName            \\
			                                        & \firmaStrasse         \\
			                                        & \firmaPlz             \\[0.2cm]
			Trainter at training company:           & \betreuerFirma        \\[0.2cm]
			Expert from the cooperative university: & \betreuerDhbw         \\[2cm]
		\end{tabular}
	\end{center}
\end{titlepage}
  % Titelseite
\newcounter{savepage}
\pagenumbering{Roman}                    % Römische Seitenzahlen
\onehalfspacing

% ------ Erklärung, Sperrvermerk, Abstact
\chapter*{Statutory declaration}
I hereby acknowledge, that my \arbeit{} with the topic:
\begin{quote}
	\textit{\titel}
\end{quote}
according to § 5 of the \enquote{Studien- und Prüfungsordnung DHBW Technik} from 29th of September 2017 has been independently written and no other sources or aids other than those specified have been used. The work has not yet been submitted to any other examination authority and has not been published.

\vspace{0.25cm}

I also assure that the submitted electronic version matches the printed version.

\vspace{1cm}

\verfassungsort, the \today \\[0.5cm]
\ifthenelse{\boolean{e-Abgabe}}
{\underline{Signed \autor}}
{\makebox[6cm]{\hrulefill}}\\
\autorReverse

\vspace*{\fill}

\section*{Equality notice}

For better readability, gender-specific duplications are not used.
\renewcommand{\abstractname}{Abstract} % Veränderter Name für das Abstract
\begin{abstract}
    \begin{addmargin}[1.5cm]{1.5cm}        % Erhöhte Ränder, für Abstract Look
        \thispagestyle{plain}                  % Seitenzahl auf der Abstract Seite

        \vspace{0.25cm}

        There are many algorithms which solve the \ac{APSP}. By far the most renowned is Floyd-Warshall algorithm. Most of the algorithms have a pretty bad runtime complexity as for example Floyd-Warshall algorithm with $O(|V|^3)$.

        \vspace{0.25cm}

        The aim of this work is to take a closer look at Johnson's algorithm. This algorithm poses an alternative to Floyd-Warshall algorithm with the goal of reducing runtime complexity.


    \end{addmargin}
\end{abstract}

% ------ Inhaltsverzeichnis
\singlespacing
\tableofcontents

% ------ Verzeichnisse
\renewcommand*{\chapterpagestyle}{plain}
\pagestyle{plain}
% \include{Inhalt/03_Verzeichnisse/formelgroessen} % Wird hier nicht benötigt
\chapter*{Abkürzungsverzeichnis}
\addcontentsline{toc}{chapter}{Abkürzungsverzeichnis} % Hinzufügen zum Inhaltsverzeichnis 

\begin{acronym}[WYSISWG] % längstes Kürzel wird verw. für den Abstand zw. Kürzel u. Text

	% Alphabetisch selbst sortieren - nicht verwendete Kürzel rausnehmen!
	\acro{APSP}{All-pairs shortest path}
	\acro{DP}{Dynamic Programming}
	\acro{SSSP}{Single source shortest path}

\end{acronym}
\setcounter{savepage}{\value{page}}


% ---- Inhalt der Arbeit
\cleardoublepage
\pagenumbering{arabic}                  % Arabische Seitenzahlen für den Hauptteil
\setlength{\parskip}{0.5\baselineskip}  % Abstand zwischen Absätzen
\rmfamily
\renewcommand*{\chapterpagestyle}{scrheadings}
\pagestyle{scrheadings}
\onehalfspacing
\chapter{Dijkstra}
Dijkstra algorithm is a greedy algorithm for finding the shortest paths between one vertex and all other vertices in a given graph.
Therefore it solves the \ac{SSSP}. One of the major constraints of Dijkstra algorithm is that it can only be used on graphs
with strictly positive edge weigths. \cite{Tamimi.2015}

Despite the mentioned limitations Dijkstra is one of the fastest algorithms to solve the \ac{SSSP}. The respective complexity depends on the implementation of the min priority queue. Using a fibonacci heap results in a runtime complexity of $O(|V|*log|V| + |E|)$. \cite{Kiruthika.2012}
\chapter{Bellman-Ford}
Bellman-Ford is the most renowned algorithm for solving the \ac{APSP} with \ac{DP}. As opposed to Dijkstra it also works with graphs which contain negative edge weights. It also is able to detect negative cycles in graphs. \cite{Kiruthika.2012, Tamimi.2015}

On the other hand the Bellman-Ford algorithm has a runtime complexity of $O(|V|*|E|)$ which is not to be neglected. \cite{Kiruthika.2012, Tamimi.2015}
\chapter{Johnson's algorithm}
Johnson's algorithm combines the previously considered algorithms Dijkstra \& Bellman-Ford to solve the \ac{APSP} with a faster runtime complexity than Floyd-Warshall algorithm, which runs in $O(|V|^3)$. \cite{Kiruthika.2012, Tamimi.2015}

To do so the limitations of Dijkstra have to be considered. As Dijkstra doesn't support negative edge weights Bellman-Ford is used to transform the graph into a graph with non-negative edge weights (or raises detected negative cycles). After that Dijkstra is applied to every vertex in the graph to solve the \ac{SSSP} for every vertex. The final shortest distances can be computed by a combination of the values computed by Bellman-Ford and Dijkstra. \cite{Kiruthika.2012}

The runtime complexity of this algorithm consists of the different stages of the algorithm. The first stage - transformation via Bellman-Ford - is the plain application of the Bellman-Ford algorithm and therefore has a runtime complexity of $O(|V|*|E|)$. The second stage - computation of shortest paths for all pairs of vertices via Dijkstra - adds a factor of $|V|$ to the runtime complexity as Dijkstra is applied to every vertex. Therefore this stage has a runtime complexity of $O(|V|^2*log|V|)$. The last stage can be computed in constant time as it just accesses previously computed values. \cite{Kiruthika.2012, Tamimi.2015} \\
Putting together the single parts Johnson's algorithm exhibits a runtime complexity of $O(|V|^2*log|V| + |V|*|E|)$, which is considerably faster than $O(|V|^3)$. Therefore we have found a faster alternative to Floyd-Warshall algorithm.

% ---- Literaturverzeichnis
\cleardoublepage
\renewcommand*{\chapterpagestyle}{plain}
\pagestyle{plain}
\pagenumbering{Roman}                   % Römische Seitenzahlen
\setcounter{page}{\numexpr\value{savepage}+1}
\printbibliography[title=Literaturverzeichnis]

% ---- Anhang
\appendix
%\clearpage
%\pagenumbering{Roman}  % römische Seitenzahlen für Anhang

\newpage
\end{document}
